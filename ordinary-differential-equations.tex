\documentclass{article}
/Users/jmft2/Documents/writing/preamble-light.tex
\title{Ordinary differential equations}
\author{T. J. Crawford, J. Goedecke, P. Haas, E. Lauga, J. Munro, J. M. F. Tsang}

\begin{document}
\maketitle

The relevant Cambridge undergraduate course is IA Differential Equations.

An ordinary differential equation expresses a relationship between an
independent variable $x$, a dependent variable $y(x)$, and derivatives $y'(x)$,
$y''(x)$, \dots. A differential equation is said to be $n$th-order if
$y^{(n)}(x)$ is the highest-order derivative that appears.

\section{Linear equations}

An $n$th-order ODE is said to be \textit{linear} if it takes the form
\begin{equation} \label{eqn:linear-ode}
    a_0(x) y + a_1(x) y' + a_2(x) y'' + \dots + a_n y^{(n)} = f(x); 
\end{equation}
that is, if $y$ and its derivatives do not appear in any nonlinear function.
The ODE is said to be \textit{homogeneous} if $f(x) = 0$, and it is said to have
\textit{constant coefficients} if each of the $a_i$ is constant. 

\subsection{Homogeneous equations}

An $n$th-order linear homogeneous equation has $n$ linearly independent
solutions $y_i$, and any solution can be written in the form
$$ y = \sum_{i=1}^n C_i y_n $$
for some constants $C_i$. Initial and boundary conditions impose conditions on
the $C_i$. Except in degenerate cases, a solution is uniquely determined when
$n$ such conditions are imposed. 

\subsection{Exponential solutions}

If the $a_i$ in \ref{eqn:linear-ode} are constants and $f = 0$, then the
linearly independent solutions $y_i$ take the form $y_i = \exp(\lambda_i x)$,
where the $\lambda_i$ are the solutions to the \textit{characteristic
polynomial} equation
$$ a_0 + a_1 \lambda + a_2 \lambda^2 + \dots + a_n \lambda^n = 0. $$

\paragraph{Repeated roots} If $\lambda$ is a root of the polynomial with multiplicity
$m>1$, then the corresponding solutions take the form $x^k \exp(\lambda x)$ for
$k=0,\dots,m-1$.

Example: Consider the ODE
$$ y''' + y'' - 5y + 3 = 0. $$
This is a homogeneous equation with constant coefficients. The characteristic
polynomial is 
$$ \lambda^3 + \lambda^2 - 5\lambda + 3 = 0 $$
which can be factorised as 
$$ (\lambda-1)^2 (\lambda+3) = 0 $$
which has roots $-3$ and $+1$; the root $\lambda=1$ has multiplicty 2. The three
linearly independent solutions to the differential equation are therefore
$$ y_1 = \exp(-3x) \text{ } y_2 = \exp(x) \text{ } y_3 = x\exp(x). $$

\subsection{Particular solutions}

\subsection{Series solutions}

If the $a_i(x)$ are polynomials in $x$, then a series solution can be sought in
the neighbourhood of a point. One makes the ansatz 
$$ y = \sum_{k=0}^\infty c_k x^k $$
and substitutes into 

\section{Other methods}

An equation of the form $\od{y}{x} = f(x)g(y)$ can be integrating by rearranging to 
$$ \frac{\dif y}{g(y)} = f(x)\dif x, $$
integrating both sides, and rearranging to get $y$ as a function of $x$.
(Sometimes, it may not be possible to do that analytically.)

\end{document}
