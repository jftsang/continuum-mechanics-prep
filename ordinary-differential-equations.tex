\documentclass{article}
/Users/jmft2/Documents/writing/preamble-light.tex
\title{Ordinary differential equations}
\author{T. J. Crawford, J. Goedecke, P. Haas, E. Lauga, J. Munro, J. M. F. Tsang}

\begin{document}
\maketitle

The relevant Cambridge undergraduate course is IA Differential Equations.

An ordinary differential equation expresses a relationship between an
independent variable $x$, a dependent variable $y(x)$, and derivatives $y'(x)$,
$y''(x)$, \dots. A differential equation is said to be $n$th-order if
$y^{(n)}(x)$ is the highest-order derivative that appears.

An $n$th-order ODE is said to be \textit{linear} if it takes the form
$$ a_0(x) y + a_1(x) y' + a_2(x) y'' + \dots + a_n y^{(n)} = f(x); $$ 
that is, if $y$ and its derivatives do not appear in any nonlinear function.
The ODE is said to be \textit{homogeneous} if $f(x) = 0$, and it is said to have
\textit{constant coefficients} if each of the $a_i$ is constant. 

An $n$th-order linear homogeneous equation has $n$ linearly independent
solutions $y_i$, and any solution can be written in the form
$$ y = \sum_{i=1}^n C_i y_n $$
for some constants $C_i$. Initial and boundary conditions impose conditions on
the $C_i$. Except in degenerate cases, a solution is uniquely determined when
$n$ such conditions are imposed. 

If an $n$th-order linear homogeneous equation has constant coefficients, then
the solutions $y_i$ take the form $y_i = \exp(\lambda_i x)$, where the
$\lambda_i$ are the solutions to the polynomial equation
$$ a_0 + a_1 \lambda + a_2 \lambda^2 + \dots + a_n \lambda^n = 0. $$

\end{document}
