\documentclass{article}
/Users/jmft2/Documents/writing/preamble-light.tex
\title{Tensors}
\author{T. J. Crawford, J. Goedecke, P. Haas, E. Lauga, J. Munro, J. M. F. Tsang}

\begin{document}
\maketitle

\section{Relevant courses}

The relevant Cambridge undergraduate course is IA Vector Calculus. 

\section{Books}


\begin{itemize}
\item K. F. Riley, M. P. Hobson and S. J. Bence \textit{Mathematical Methods for
Physics and Engineering}. Cambridge University Press 2002.
\item D. E. Bourne and P. C. Kendall \textit{Vector Analysis and Cartesian Tensors}.
3rd edition, Nelson Thornes 1999
\end{itemize}

\section{Notes}

\subsection{Definition and examples}

Consider orthogonal right-handed bases $\{\bs{e}_i\}$ and $\{\bs{e}'_i\}$ in
$\reals^3$ with corresponding Cartesian coordinates $\{x_i\}$ and $\{x'_i\}$.
Then a vector $\bs{x}\in\reals^3$ can be written as
$$ \bs{x} = x_i\bs{e}_i = x'_i\bs{e}'_i $$
(using summation convention throughout these notes).

These two bases are related by a rotation: 
$$ \bs{e}'_i = R_{ip} \bs{e}_p \text{ and } x'_i = R_{ip} x_p $$
where $R$ is a rotation matrix, so
\begin{itemize}
\item $R$ is orthogonal: $R_{ip}R_{jp} = R_{qi}R_{qj} = \delta_{ij} $, and 
\item $\det R = 1$.
\end{itemize}

Tensors are geometrical objects which obey a generalised form of this
transformation rule. By definition, a tensor $T$ of rank $n$ has components
$T_{ij\dots k}$ (with $n$ indices) with respect to each basis $\{\bs{e}_i\}$
or coordinate system $\{x_i\}$, obeying the \textit{tensor
transformation rule}
$$ T'_{ij\dots k} = R_{ip}R_{jq}\dots R_{kr} T_{pq\dots r}. $$
under a change of basis.

\paragraph{Examples} Here are some examples of the tensor transformation rule for different ranks:
\begin{itemize}
    \item Rank 0: $T' = T$. Rank 0 tensors are \textit{scalars}. 
    \item Rank 1: $T'_i = R_{ip} T_p$. As we have seen, these are \textit{vectors}.
    \item Rank 2: $T'_{ij} = R_{ip}R_{jq} T_{pq}$. These are \textit{matrices} which represent \textit{linear maps} or \textit{quadratic forms}.
\end{itemize}

\paragraph{Special cases} The tensors $\delta_{ij}$ and $\epsilon_{ijk}$ are
tensors of rank 2 and 3 respectively, with the special property that their
components are unchanged under any change of basis:
$$ R_{ip} R_{jq} \delta_{pq} = R_{ip}R_{jp} = \delta_{ij} $$
and 
$$ R_{ip} R_{jq} R_{kr} \epsilon_{pqr} = (\det R)\epsilon_{ijk} = \epsilon_{ijk}. $$

\paragraph{Symmetric and antisymmetric tensors} A tensor of rank $n$ obeying 
$T_{ijp\dots q} = \pm T_{jip\dots q}$ is said to be
\textit{symmetric/antisymmetric} in the indices $i$ and $j$. A tensor is said to
be \textit{totally symmetric/antisymmetric} if it is symmetric/antisymmetric
under any such swap of indices.

So, $\delta_{ij}$ is totally symmetric and $\epsilon_{ijk}$ is totally
antisymmetric.

In $\reals^3$, any totally antisymmetric tensor of rank 3 takes the form
$T_{ijk} = \lambda \epsilon_{ijk}$ for some $\lambda$. There is no totally
antisymmetric tensor of rank $n>3$ (unless all components are zeros).

\section{Exercises}

\subsection{Basic properties}

Show each of the following:
\begin{itemize}
\item If $T$ and $S$ are both tensors of rank $n$, then $(T+S)_{ij\dots k} = T_{ij\dots k} + S_{ij\dots k}$. (Hint: Use the transformation rule.)
\item If $\alpha$ is a scalar, then $(\alpha T)_{ij\dots k} = \alpha T_{ij\dots k}$.
\item Hence, any linear combination of rank $n$ tensors is itself a rank $n$ tensor.
\item If $T$ and $S$ are tensors of rank $n$ and $m$ respectively, then the
\textit{tensor product} 
$$ (T\otimes S)_{ij\dots kpq\dots r} = T_{ij\dots k} S_{pq\dots r} $$
is a tensor of rank $n+m$.
\item If $\bs{u}$, $\bs{v}$, \dots, $\bs{w}$ are $n$ vectors, then $T_{ij\dots k} = u_i v_j \dots w_k $ defines a tensor of rank $n$.
\item If $T_{ijp\dots q}$ is a tensor of rank $n$, then $S_{p\dots q} =
\delta_{ij} T_{ijp\dots q}$ is a tensor of rank $n-2$. (Note that contracting on
a different pair of indices in general results in a different tensor.)
\item The trace of a matrix, $T_{ii}$, is a rank 0 tensor.
\end{itemize}

\subsection{Further exercises}

Let $u_i(\bs{x})$ be a vector field and let $\sigma_{ij}(\bs{x})$ be a
second-rank tensor field. Show that:
\begin{itemize}
\item $\partial u_i/\partial x_j$ transforms as a rank 2 tensor.
\item $\divg\bs{u} = \partial u_i/\partial x_i$ is a scalar.
\item $\partial \sigma_{ij}/partial x_j$ transforms as a vector.
\end{itemize}

Let $T$ be a tensor of rank 3, satisfying
$$ T_{ijk} = T_{jik} \text{ and } T_{ijk} = -T_{ikj}. $$
Show that $T_{ijk} = 0$.

Let $T$ be a tensor of rank 4, satisfying 
$$ T_{ijkl} = -T_{jikl} = -T_{ijlk} \text{ and } T_{ijij} = 0. $$
Show that
$$ T_{ijkl} = \epsilon_{ijp}\epsilon_{klq} S_{pq} $$
where $ S_{pq} = -T_{rqrp} $.

\end{document}
