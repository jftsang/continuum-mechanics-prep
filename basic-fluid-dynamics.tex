\documentclass{article}
% Jonny's lightweight preamble. 

% Usage: 
% \documentclass[a4paper,11pt]{article}
% % Jonny's lightweight preamble. 

% Usage: 
% \documentclass[a4paper,11pt]{article}
% % Jonny's lightweight preamble. 

% Usage: 
% \documentclass[a4paper,11pt]{article}
% \input{preamble-light.tex}

%%% BEGINNING OF PREAMBLE 

\usepackage{amsmath,amsthm,amssymb,amsfonts,gensymb}

\usepackage[a4paper, margin=2cm]{geometry}

\usepackage{parskip}

\usepackage{hyperref}
\usepackage{bm} % Nicer bolds for vectors (including bold Greek letters and bold grad symbol)
\usepackage{verbatim} % Include files verbatim using \verbatiminput

\usepackage{graphicx}
\usepackage{rotating}
\usepackage{subfigure}

% commath: common maths things
% http://anorien.csc.warwick.ac.uk/mirrors/CTAN/macros/latex/contrib/commath/commath.pdf 
\usepackage{commath}

\usepackage[final]{pdfpages}
\usepackage[autostyle]{csquotes}

% Some shortcuts for vector calculus. 
\newcommand{\bs}{\boldsymbol}
\newcommand{\grad}{\bs{\nabla}}
\newcommand{\cross}{\times}
\newcommand{\curl}{\grad\cross}
\newcommand{\divg}{\grad\cdot}
\newcommand{\DDtfull}{\left(\frac{\partial}{\partial t} + \bs{u}\cdot\grad\right)}
\newcommand{\DDt} [1] {\frac{\mathrm{D}#1}{\mathrm{D}t}}
\newcommand{\dDDt} [1] {\displaystyle\frac{\mathrm{D}#1}{\mathrm{D}t}} 

% Asterisks: \ask
%%% See http://tex.stackexchange.com/questions/59183/normal-high-asterisk-in-equation-mode
%%% magic code starts
\mathcode`*=\string"8000
\begingroup
\catcode`*=\active
\xdef*{\noexpand\textup{\string*}}
\endgroup
%%% magic code ends

% Mathematical writing
\newcommand{\theorem}{\paragraph{Theorem}}
\newcommand \theoremp[1] {\paragraph{Theorem (#1)}}
\newcommand{\proposition}{\paragraph{Proposition}}
\renewcommand{\proof}{\textit{Proof. }}
\newcommand{\example}{\paragraph{Example}}
\newcommand{\definition}{\paragraph{Definition}}
\newcommand{\exercise}{\paragraph{Exercise}}
\newcommand{\exercises}{\paragraph{Exercises}}
\newcommand{\note}{\textit{Note. }}
\newcommand{\notes}{\textit{Notes. }}

% Shortcuts for some mathematical functions
\newcommand{\sgn}{\mathrm{sgn}}
\newcommand{\erf}{\mathrm{erf}}
\newcommand{\erfc}{\mathrm{erfc}}

\newcommand \naturals {\mathbb{N}}
\newcommand \integers {\mathbb{Z}}
\newcommand \rationals {\mathbb{Q}}
\newcommand \reals {\mathbb{R}}
\newcommand \complex {\mathbb{C}}
\newcommand \Rey {\mathrm{Re}} % Reynolds number
\renewcommand \Pr {\mathbb{P}} % probability
\newcommand \E {\mathbb{E}} % expectation

%%% Column vectors: see http://tex.stackexchange.com/questions/2705/typesetting-column-vector
\newcount\colveccount
\newcommand*\colvec[1]{
        \global\colveccount#1
        \begin{pmatrix}
        \colvecnext
}
\def\colvecnext#1{
        #1
        \global\advance\colveccount-1
        \ifnum\colveccount>0
                \\
                \expandafter\colvecnext
        \else
                \end{pmatrix}
        \fi
}

\usepackage{centernot}
\newcommand{\nimplies} {\centernot\implies}

\usepackage{multicol}
%%% END OF PREAMBLE


%%% BEGINNING OF PREAMBLE 

\usepackage{amsmath,amsthm,amssymb,amsfonts,gensymb}

\usepackage[a4paper, margin=2cm]{geometry}

\usepackage{parskip}

\usepackage{hyperref}
\usepackage{bm} % Nicer bolds for vectors (including bold Greek letters and bold grad symbol)
\usepackage{verbatim} % Include files verbatim using \verbatiminput

\usepackage{graphicx}
\usepackage{rotating}
\usepackage{subfigure}

% commath: common maths things
% http://anorien.csc.warwick.ac.uk/mirrors/CTAN/macros/latex/contrib/commath/commath.pdf 
\usepackage{commath}

\usepackage[final]{pdfpages}
\usepackage[autostyle]{csquotes}

% Some shortcuts for vector calculus. 
\newcommand{\bs}{\boldsymbol}
\newcommand{\grad}{\bs{\nabla}}
\newcommand{\cross}{\times}
\newcommand{\curl}{\grad\cross}
\newcommand{\divg}{\grad\cdot}
\newcommand{\DDtfull}{\left(\frac{\partial}{\partial t} + \bs{u}\cdot\grad\right)}
\newcommand{\DDt} [1] {\frac{\mathrm{D}#1}{\mathrm{D}t}}
\newcommand{\dDDt} [1] {\displaystyle\frac{\mathrm{D}#1}{\mathrm{D}t}} 

% Asterisks: \ask
%%% See http://tex.stackexchange.com/questions/59183/normal-high-asterisk-in-equation-mode
%%% magic code starts
\mathcode`*=\string"8000
\begingroup
\catcode`*=\active
\xdef*{\noexpand\textup{\string*}}
\endgroup
%%% magic code ends

% Mathematical writing
\newcommand{\theorem}{\paragraph{Theorem}}
\newcommand \theoremp[1] {\paragraph{Theorem (#1)}}
\newcommand{\proposition}{\paragraph{Proposition}}
\renewcommand{\proof}{\textit{Proof. }}
\newcommand{\example}{\paragraph{Example}}
\newcommand{\definition}{\paragraph{Definition}}
\newcommand{\exercise}{\paragraph{Exercise}}
\newcommand{\exercises}{\paragraph{Exercises}}
\newcommand{\note}{\textit{Note. }}
\newcommand{\notes}{\textit{Notes. }}

% Shortcuts for some mathematical functions
\newcommand{\sgn}{\mathrm{sgn}}
\newcommand{\erf}{\mathrm{erf}}
\newcommand{\erfc}{\mathrm{erfc}}

\newcommand \naturals {\mathbb{N}}
\newcommand \integers {\mathbb{Z}}
\newcommand \rationals {\mathbb{Q}}
\newcommand \reals {\mathbb{R}}
\newcommand \complex {\mathbb{C}}
\newcommand \Rey {\mathrm{Re}} % Reynolds number
\renewcommand \Pr {\mathbb{P}} % probability
\newcommand \E {\mathbb{E}} % expectation

%%% Column vectors: see http://tex.stackexchange.com/questions/2705/typesetting-column-vector
\newcount\colveccount
\newcommand*\colvec[1]{
        \global\colveccount#1
        \begin{pmatrix}
        \colvecnext
}
\def\colvecnext#1{
        #1
        \global\advance\colveccount-1
        \ifnum\colveccount>0
                \\
                \expandafter\colvecnext
        \else
                \end{pmatrix}
        \fi
}

\usepackage{centernot}
\newcommand{\nimplies} {\centernot\implies}

\usepackage{multicol}
%%% END OF PREAMBLE


%%% BEGINNING OF PREAMBLE 

\usepackage{amsmath,amsthm,amssymb,amsfonts,gensymb}

\usepackage[a4paper, margin=2cm]{geometry}

\usepackage{parskip}

\usepackage{hyperref}
\usepackage{bm} % Nicer bolds for vectors (including bold Greek letters and bold grad symbol)
\usepackage{verbatim} % Include files verbatim using \verbatiminput

\usepackage{graphicx}
\usepackage{rotating}
\usepackage{subfigure}

% commath: common maths things
% http://anorien.csc.warwick.ac.uk/mirrors/CTAN/macros/latex/contrib/commath/commath.pdf 
\usepackage{commath}

\usepackage[final]{pdfpages}
\usepackage[autostyle]{csquotes}

% Some shortcuts for vector calculus. 
\newcommand{\bs}{\boldsymbol}
\newcommand{\grad}{\bs{\nabla}}
\newcommand{\cross}{\times}
\newcommand{\curl}{\grad\cross}
\newcommand{\divg}{\grad\cdot}
\newcommand{\DDtfull}{\left(\frac{\partial}{\partial t} + \bs{u}\cdot\grad\right)}
\newcommand{\DDt} [1] {\frac{\mathrm{D}#1}{\mathrm{D}t}}
\newcommand{\dDDt} [1] {\displaystyle\frac{\mathrm{D}#1}{\mathrm{D}t}} 

% Asterisks: \ask
%%% See http://tex.stackexchange.com/questions/59183/normal-high-asterisk-in-equation-mode
%%% magic code starts
\mathcode`*=\string"8000
\begingroup
\catcode`*=\active
\xdef*{\noexpand\textup{\string*}}
\endgroup
%%% magic code ends

% Mathematical writing
\newcommand{\theorem}{\paragraph{Theorem}}
\newcommand \theoremp[1] {\paragraph{Theorem (#1)}}
\newcommand{\proposition}{\paragraph{Proposition}}
\renewcommand{\proof}{\textit{Proof. }}
\newcommand{\example}{\paragraph{Example}}
\newcommand{\definition}{\paragraph{Definition}}
\newcommand{\exercise}{\paragraph{Exercise}}
\newcommand{\exercises}{\paragraph{Exercises}}
\newcommand{\note}{\textit{Note. }}
\newcommand{\notes}{\textit{Notes. }}

% Shortcuts for some mathematical functions
\newcommand{\sgn}{\mathrm{sgn}}
\newcommand{\erf}{\mathrm{erf}}
\newcommand{\erfc}{\mathrm{erfc}}

\newcommand \naturals {\mathbb{N}}
\newcommand \integers {\mathbb{Z}}
\newcommand \rationals {\mathbb{Q}}
\newcommand \reals {\mathbb{R}}
\newcommand \complex {\mathbb{C}}
\newcommand \Rey {\mathrm{Re}} % Reynolds number
\renewcommand \Pr {\mathbb{P}} % probability
\newcommand \E {\mathbb{E}} % expectation

%%% Column vectors: see http://tex.stackexchange.com/questions/2705/typesetting-column-vector
\newcount\colveccount
\newcommand*\colvec[1]{
        \global\colveccount#1
        \begin{pmatrix}
        \colvecnext
}
\def\colvecnext#1{
        #1
        \global\advance\colveccount-1
        \ifnum\colveccount>0
                \\
                \expandafter\colvecnext
        \else
                \end{pmatrix}
        \fi
}

\usepackage{centernot}
\newcommand{\nimplies} {\centernot\implies}

\usepackage{multicol}
%%% END OF PREAMBLE

\title{Basic Fluid Dynamics}
\author{T. J. Crawford, J. Goedecke, P. Haas, E. Lauga, J. R. Lister, J. Munro, J. M. F. Tsang}

\begin{document}
\maketitle

\section{Relevant courses}

The relevant Cambridge undergraduate courses are IB Fluid Dynamics, II Fluid
Dynamics and II Waves. 

\section{Books}

\section{Notes}

\subsection{Lagrange, Euler and the material derivative}

There are two natural ways to describe a fluid flow. In the \textit{Eulerian
view}, one considers the evolution of fluid properties (such as velocity and
pressure) at a fixed point, while in the \textit{Lagrangian view}, one follows
the path of a certain fluid element and considers what happens to that element.

If the velocity field is $\bs{u}(\bs{x},t)$, the \textit{pathline} of a fluid
particle is the path traced out by that particle; it is the solution to 
$$ \dod{\bs{x}}{t} = \bs{u}(\bs{x},t). $$
This is useful in the Lagrangian description.

However, it is usually easier to work under the Eulerian view, and to take a
`snapshot' of fluid properties at all places at a given instant in time. A
\textit{streamline} is a curve that is instantaneously everywhere parallel to
the velocity field. 

For a time-independent fluid flow, streamlines and pathlines coincide with each
other, but this is not true for a time-dependent flow.

Consider a quantity $F(\bs{x},t)$.  What is the rate of change
of $F$ following the fluid particle at $(\bs{x},t)$? Since the local velocity is
$\bs{u}(\bs{x},t)$, the particle will move by a distance
$\delta\bs{x}=\bs{u}(x,t) \delta t$ over the time $\delta t$. Hence the change
in $F$ is
$$ \delta F = F(\bs{x}+\delta\bs{x},t+\delta t) - F(\bs{x},t)
     = (\delta\bs{x}\cdot\grad)F + \delta t \dpd{F}{t} + \cdots $$
and we define 
$$ \DDt{F}{t} = \lim_{\delta t\rightarrow0} 
 \frac{F(\bs{x}+\delta\bs{x},t+\delta t) - F(\bs{x},t)}{\delta t} 
 = \dpd{F}{t} + \bs{u}\cdot\grad F. $$
We call $\DDt{F}{t}$ the \textit{material derivative} or \textit{Lagrangian
derivative} of $F$, and we contrast it against $\pd{F}{t}$, which refers to the rate
of change of $F$ at a fixed position.

\subsection{Governing equations}

Mass conservation: 
$$ \dpd{\rho}{t} + \divg(\rho\bs{u}) = 0 $$

Momentum balance:
$$ \dpd{}{t}(\rho\bs{u}) + \divg(\rho\bs{u}\bs{u}) = \divg\bs{\sigma} + \bs{f} $$
or, using the mass conservation equation,
$$ \rho\left(\dpd{\bs{u}}{t} + \bs{u}\cdot\grad\bs{u}\right) = \divg\bs{\sigma} + \bs{f} $$
Here, $\bs{f}$ represents a \textit{body force}, such as gravity or a Coriolis
force.  And $\sigma$ is the \textit{Cauchy stress tensor} and describes the
effects of \textit{surface forces}. If $\dif S$ is a small surface element with
normal $\bs{n}$, then the force on that surface element by the fluid is
$\dif\bs{F}=\bs{\sigma}\cdot\bs{n}\dif S$, and the \textit{stress} (force per
unit area) is $\bs{\sigma}\cdot\bs{n}$.

Energy, assuming incompressibility:
$$ \dpd{}{t} \left(\frac{1}{2}\rho|\bs{u}|^2\right) 
   + \bs{u}\cdot\grad\left(\frac{1}{2}\rho|\bs{u}|^2\right) 
   = \divg(\bs{u}\cdot\bs{\sigma}) + \bs{u}\cdot\bs{f} - \bs{\sigma}:\bs{e} $$
where $\bs{e}$ is the \textit{rate of strain tensor}. The last term,
$\bs{\sigma}:\bs{e}$, represents \textit{dissipation} and is usually written as
$\Delta$.

One gets equations for mass and momentum conservation by applying the divergence
theorem on a small volume. A similar argument on angular momentum shows that,
assuming that long-range forces exert no couple on individual fluid elements,
the stress tensor must be symmetric: $\sigma_{ij}=\sigma_{ji}$.

The above holds for all incompressible fluids. To close the system, we need to
provide a relationship between $\bs{\sigma}$ and the other quantities. Such a
relationship is a \textit{constitutive equation}, and is determined by the
nature of the material.

An incompressible Newtonian fluid satisfies $\DDt{\rho}=\divg\bs{u}=0$ and has
the constitutive equation
$$ \bs{\sigma} = -p \bs{I} + 2\mu\bs{e}, $$
where $p$ is the \textit{pressure}. We do not need to provide another equation
to specify $p$ and close the system; $p$ is determined by the condition that
$\divg\bs{u}=0$.\footnote{In some sense, $p$ acts as a Lagrange multiplier which
imposes the incompressibility constraint.} The strain rate tensor is given by
$$ \bs{e} = \frac{1}{2}(\grad\bs{u}+\grad\bs{u}^T) $$
so that $\trace\bs{e} = 0$. And the dissipation rate is given by
$$\Delta = \bs{\sigma}:\bs{e}=2\mu\bs{e}:\bs{e}.$$

The assumptions that give rise to this constitutive relation are that the the
deviatoric stress (i.e. the anisotropic part of $\bs{\sigma}$) is
\textit{linear} in $\grad\bs{u}$, and \textit{instantaneous} (i.e. there is no
dependence on the history of deformation). 

After all this, one gets the \textit{Navier-Stokes equations}:
$$ \rho\DDt{\bs{u}} = -\grad p + \mu\grad^2 \bs{u} $$
and 
$$ \divg\bs{u} = 0. $$

\subsection{Boundary conditions}

\paragraph{Kinematic boundary conditions}  At a fluid-fluid interface,
$\bs{u}\cdot\bs{n}$ is continuous (by mass conservation) and
$\bs{u}\cross\bs{n}$ is continuous (to avoid infinite stresses). 

At a rigid boundary moving with velocity $\bs{U}$, the kinematic BCs simplify to
$$ u = U. $$

For a surface given by $z=h(x,y,t)$, the kinematic BCs can be expressed as
$$ \DDt{}(h-z) = 0 $$
on $z=h$.

\paragraph{Dynamic boundary conditions} In the absence of surface tension,
$\bs{\sigma}\cdot\bs{n}$ is continuous. With surface tension,
$$ [\bs{\sigma}\cdot\bs{n}] = \gamma\kappa\bs{n} - \grad_s\gamma $$
where $\gamma$ is the \textit{coefficient of surface tension}, $\grad_s$ is the
gradient on the surface, and $\kappa = \grad_s\cdot\bs{n}$ is the
\textit{total curvature} of the surface.

\section{Exercises}

\end{document}
