\documentclass{article}
/Users/jmft2/Documents/writing/preamble-light.tex
\title{Preparatory materials for Part III Continuum Mechanics}
\author{T. J. Crawford, J. Goedecke, P. Haas, E. Lauga, J. Munro, J. M. F. Tsang}

\begin{document}
\maketitle

These notes are meant to be an overview of the mathematical methods and fluid
dynamical ideas that you need to have before taking continuum mechanics courses
in Part III. While not everything here is \textit{central} to every course, you
should at least be familiar with the concepts here.
These notes are not intended to be complete. Please refer to the lecture notes
of the relevant undergraduate courses in the Mathematical Tripos. Appropriate
textbooks are also given in the schedules. 

\section{Mathematical methods}
\begin{itemize}
    \item Elementary linear algebra
    \item Ordinary differential equations
    \item Index notation
    \item Cylindrical and spherical coordinates 
    \item Vector calculus
    \item Integral theorems 
    \item Tensors 
    \item Solution of PDEs by separation of variables 
    \item Fourier and Laplace transforms 
    \item Complex variable techniques 
    \item Scaling analysis 
    \item Elementary asymptotic methods 
\end{itemize}

\section{Fluid dynamics}
\begin{itemize}
    \item Kinematics 
    \item Navier-Stokes equations 
    \item Couette and Poiseuille flows 
    \item Reynolds number 
    \item Thin-layer flows 
    \item Rotating frames 
    \item Compressible fluids 
    \item Dispersive waves 
    \item Nonlinear waves 
\end{itemize}
\end{document}
