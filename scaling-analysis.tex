\documentclass{article}
% Jonny's lightweight preamble. 

% Usage: 
% \documentclass[a4paper,11pt]{article}
% % Jonny's lightweight preamble. 

% Usage: 
% \documentclass[a4paper,11pt]{article}
% % Jonny's lightweight preamble. 

% Usage: 
% \documentclass[a4paper,11pt]{article}
% \input{preamble-light.tex}

%%% BEGINNING OF PREAMBLE 

\usepackage{amsmath,amsthm,amssymb,amsfonts,gensymb}

\usepackage[a4paper, margin=2cm]{geometry}

\usepackage{parskip}

\usepackage{hyperref}
\usepackage{bm} % Nicer bolds for vectors (including bold Greek letters and bold grad symbol)
\usepackage{verbatim} % Include files verbatim using \verbatiminput

\usepackage{graphicx}
\usepackage{rotating}
\usepackage{subfigure}

% commath: common maths things
% http://anorien.csc.warwick.ac.uk/mirrors/CTAN/macros/latex/contrib/commath/commath.pdf 
\usepackage{commath}

\usepackage[final]{pdfpages}
\usepackage[autostyle]{csquotes}

% Some shortcuts for vector calculus. 
\newcommand{\bs}{\boldsymbol}
\newcommand{\grad}{\bs{\nabla}}
\newcommand{\cross}{\times}
\newcommand{\curl}{\grad\cross}
\newcommand{\divg}{\grad\cdot}
\newcommand{\DDtfull}{\left(\frac{\partial}{\partial t} + \bs{u}\cdot\grad\right)}
\newcommand{\DDt} [1] {\frac{\mathrm{D}#1}{\mathrm{D}t}}
\newcommand{\dDDt} [1] {\displaystyle\frac{\mathrm{D}#1}{\mathrm{D}t}} 

% Asterisks: \ask
%%% See http://tex.stackexchange.com/questions/59183/normal-high-asterisk-in-equation-mode
%%% magic code starts
\mathcode`*=\string"8000
\begingroup
\catcode`*=\active
\xdef*{\noexpand\textup{\string*}}
\endgroup
%%% magic code ends

% Mathematical writing
\newcommand{\theorem}{\paragraph{Theorem}}
\newcommand \theoremp[1] {\paragraph{Theorem (#1)}}
\newcommand{\proposition}{\paragraph{Proposition}}
\renewcommand{\proof}{\textit{Proof. }}
\newcommand{\example}{\paragraph{Example}}
\newcommand{\definition}{\paragraph{Definition}}
\newcommand{\exercise}{\paragraph{Exercise}}
\newcommand{\exercises}{\paragraph{Exercises}}
\newcommand{\note}{\textit{Note. }}
\newcommand{\notes}{\textit{Notes. }}

% Shortcuts for some mathematical functions
\newcommand{\sgn}{\mathrm{sgn}}
\newcommand{\erf}{\mathrm{erf}}
\newcommand{\erfc}{\mathrm{erfc}}

\newcommand \naturals {\mathbb{N}}
\newcommand \integers {\mathbb{Z}}
\newcommand \rationals {\mathbb{Q}}
\newcommand \reals {\mathbb{R}}
\newcommand \complex {\mathbb{C}}
\newcommand \Rey {\mathrm{Re}} % Reynolds number
\renewcommand \Pr {\mathbb{P}} % probability
\newcommand \E {\mathbb{E}} % expectation

%%% Column vectors: see http://tex.stackexchange.com/questions/2705/typesetting-column-vector
\newcount\colveccount
\newcommand*\colvec[1]{
        \global\colveccount#1
        \begin{pmatrix}
        \colvecnext
}
\def\colvecnext#1{
        #1
        \global\advance\colveccount-1
        \ifnum\colveccount>0
                \\
                \expandafter\colvecnext
        \else
                \end{pmatrix}
        \fi
}

\usepackage{centernot}
\newcommand{\nimplies} {\centernot\implies}

\usepackage{multicol}
%%% END OF PREAMBLE


%%% BEGINNING OF PREAMBLE 

\usepackage{amsmath,amsthm,amssymb,amsfonts,gensymb}

\usepackage[a4paper, margin=2cm]{geometry}

\usepackage{parskip}

\usepackage{hyperref}
\usepackage{bm} % Nicer bolds for vectors (including bold Greek letters and bold grad symbol)
\usepackage{verbatim} % Include files verbatim using \verbatiminput

\usepackage{graphicx}
\usepackage{rotating}
\usepackage{subfigure}

% commath: common maths things
% http://anorien.csc.warwick.ac.uk/mirrors/CTAN/macros/latex/contrib/commath/commath.pdf 
\usepackage{commath}

\usepackage[final]{pdfpages}
\usepackage[autostyle]{csquotes}

% Some shortcuts for vector calculus. 
\newcommand{\bs}{\boldsymbol}
\newcommand{\grad}{\bs{\nabla}}
\newcommand{\cross}{\times}
\newcommand{\curl}{\grad\cross}
\newcommand{\divg}{\grad\cdot}
\newcommand{\DDtfull}{\left(\frac{\partial}{\partial t} + \bs{u}\cdot\grad\right)}
\newcommand{\DDt} [1] {\frac{\mathrm{D}#1}{\mathrm{D}t}}
\newcommand{\dDDt} [1] {\displaystyle\frac{\mathrm{D}#1}{\mathrm{D}t}} 

% Asterisks: \ask
%%% See http://tex.stackexchange.com/questions/59183/normal-high-asterisk-in-equation-mode
%%% magic code starts
\mathcode`*=\string"8000
\begingroup
\catcode`*=\active
\xdef*{\noexpand\textup{\string*}}
\endgroup
%%% magic code ends

% Mathematical writing
\newcommand{\theorem}{\paragraph{Theorem}}
\newcommand \theoremp[1] {\paragraph{Theorem (#1)}}
\newcommand{\proposition}{\paragraph{Proposition}}
\renewcommand{\proof}{\textit{Proof. }}
\newcommand{\example}{\paragraph{Example}}
\newcommand{\definition}{\paragraph{Definition}}
\newcommand{\exercise}{\paragraph{Exercise}}
\newcommand{\exercises}{\paragraph{Exercises}}
\newcommand{\note}{\textit{Note. }}
\newcommand{\notes}{\textit{Notes. }}

% Shortcuts for some mathematical functions
\newcommand{\sgn}{\mathrm{sgn}}
\newcommand{\erf}{\mathrm{erf}}
\newcommand{\erfc}{\mathrm{erfc}}

\newcommand \naturals {\mathbb{N}}
\newcommand \integers {\mathbb{Z}}
\newcommand \rationals {\mathbb{Q}}
\newcommand \reals {\mathbb{R}}
\newcommand \complex {\mathbb{C}}
\newcommand \Rey {\mathrm{Re}} % Reynolds number
\renewcommand \Pr {\mathbb{P}} % probability
\newcommand \E {\mathbb{E}} % expectation

%%% Column vectors: see http://tex.stackexchange.com/questions/2705/typesetting-column-vector
\newcount\colveccount
\newcommand*\colvec[1]{
        \global\colveccount#1
        \begin{pmatrix}
        \colvecnext
}
\def\colvecnext#1{
        #1
        \global\advance\colveccount-1
        \ifnum\colveccount>0
                \\
                \expandafter\colvecnext
        \else
                \end{pmatrix}
        \fi
}

\usepackage{centernot}
\newcommand{\nimplies} {\centernot\implies}

\usepackage{multicol}
%%% END OF PREAMBLE


%%% BEGINNING OF PREAMBLE 

\usepackage{amsmath,amsthm,amssymb,amsfonts,gensymb}

\usepackage[a4paper, margin=2cm]{geometry}

\usepackage{parskip}

\usepackage{hyperref}
\usepackage{bm} % Nicer bolds for vectors (including bold Greek letters and bold grad symbol)
\usepackage{verbatim} % Include files verbatim using \verbatiminput

\usepackage{graphicx}
\usepackage{rotating}
\usepackage{subfigure}

% commath: common maths things
% http://anorien.csc.warwick.ac.uk/mirrors/CTAN/macros/latex/contrib/commath/commath.pdf 
\usepackage{commath}

\usepackage[final]{pdfpages}
\usepackage[autostyle]{csquotes}

% Some shortcuts for vector calculus. 
\newcommand{\bs}{\boldsymbol}
\newcommand{\grad}{\bs{\nabla}}
\newcommand{\cross}{\times}
\newcommand{\curl}{\grad\cross}
\newcommand{\divg}{\grad\cdot}
\newcommand{\DDtfull}{\left(\frac{\partial}{\partial t} + \bs{u}\cdot\grad\right)}
\newcommand{\DDt} [1] {\frac{\mathrm{D}#1}{\mathrm{D}t}}
\newcommand{\dDDt} [1] {\displaystyle\frac{\mathrm{D}#1}{\mathrm{D}t}} 

% Asterisks: \ask
%%% See http://tex.stackexchange.com/questions/59183/normal-high-asterisk-in-equation-mode
%%% magic code starts
\mathcode`*=\string"8000
\begingroup
\catcode`*=\active
\xdef*{\noexpand\textup{\string*}}
\endgroup
%%% magic code ends

% Mathematical writing
\newcommand{\theorem}{\paragraph{Theorem}}
\newcommand \theoremp[1] {\paragraph{Theorem (#1)}}
\newcommand{\proposition}{\paragraph{Proposition}}
\renewcommand{\proof}{\textit{Proof. }}
\newcommand{\example}{\paragraph{Example}}
\newcommand{\definition}{\paragraph{Definition}}
\newcommand{\exercise}{\paragraph{Exercise}}
\newcommand{\exercises}{\paragraph{Exercises}}
\newcommand{\note}{\textit{Note. }}
\newcommand{\notes}{\textit{Notes. }}

% Shortcuts for some mathematical functions
\newcommand{\sgn}{\mathrm{sgn}}
\newcommand{\erf}{\mathrm{erf}}
\newcommand{\erfc}{\mathrm{erfc}}

\newcommand \naturals {\mathbb{N}}
\newcommand \integers {\mathbb{Z}}
\newcommand \rationals {\mathbb{Q}}
\newcommand \reals {\mathbb{R}}
\newcommand \complex {\mathbb{C}}
\newcommand \Rey {\mathrm{Re}} % Reynolds number
\renewcommand \Pr {\mathbb{P}} % probability
\newcommand \E {\mathbb{E}} % expectation

%%% Column vectors: see http://tex.stackexchange.com/questions/2705/typesetting-column-vector
\newcount\colveccount
\newcommand*\colvec[1]{
        \global\colveccount#1
        \begin{pmatrix}
        \colvecnext
}
\def\colvecnext#1{
        #1
        \global\advance\colveccount-1
        \ifnum\colveccount>0
                \\
                \expandafter\colvecnext
        \else
                \end{pmatrix}
        \fi
}

\usepackage{centernot}
\newcommand{\nimplies} {\centernot\implies}

\usepackage{multicol}
%%% END OF PREAMBLE

\title{Scaling analysis}
\author{T. J. Crawford, J. Goedecke, P. Haas, E. Lauga, J. Munro, J. M. F. Tsang}

\begin{document}
\maketitle

\section{Relevant courses}

The relevant Cambridge undergraduate courses are IB Fluid Dynamics, II Fluid
Dynamics and II Mathematical Biology.

\section{Books}

The principles of scaling analysis, and a number of examples, are given in
\textit{Elementary Fluid Dynamics} by D. J. Acheson, OUP 1990.

\section{Notes}

\subsection{The heat equation}

Consider the heat equation
$$ \dpd{u}{t} = \alpha\grad^2 u  $$
where $\alpha$ is the thermal diffusivity (related to the thermal conductivity
of a material $k$ by $\alpha = k/(c_p \rho)$. 

Specifically, consider this equation on the one-dimensional domain
$x\in[0,L]$ for $t>0$ (so that $\grad^2 = \od[2]{}{x}$). The initial and boundary conditions are 
$$ u(x,0) = f(x) \text{ for some given $f$} $$
and
$$ u(0,t) = u(L,t) \text{ for all t.} $$

What is the timescale for heat to decay? 

There are two approaches for analysing this problem.

\paragraph{Approach 1: Exact solution} We solve the equation exactly using
separation of variables. Looking for $u(x,t)$ as a superposition of solutions of
the form $T(t)X(x)$, we find that the solution takes the form 
$$ u(x,t) = \sum_{n=1}^\infty D_n \sin\left(\frac{n\pi x}{L}\right)
\exp\left(\frac{-n^2 \pi^2 \alpha t}{L^2}\right). $$
Therefore, the decay happens over a timescale $T \propto L^2/\alpha$.

\paragraph{Approach 2: Dimensional/scaling analysis} Let us write $u(x,t) = U \hat
u(x,t) $, where $U$ is a `typical' temperature. Also, write $t=T\hat{t}$ and
$x=L\hat{x}$. We are interested in finding the timescale $T$. The hatted
quantities are all nondimensional.

Substituting these into the heat equation, and dropping hats, one gets
$$ \frac{U}{T} u_t = \frac{\alpha U}{L^2} u_{xx}. $$
Since the dimensions must match up, 
$$ \frac{U}{T} \propto \frac{\alpha U}{L^2} $$
and so $T \propto L^2/\alpha$ as before.

\subsection{The Reynolds number}

Consider the Navier-Stokes equations
$$ \rho\DDt{\bs{u}} = -\grad p + \mu \grad^2 \bs{u} \text{ and } \divg\bs{u}=0 $$
supposing that there are no body forces.

Let $u = U\hat{u}$, $x = L\hat{x}$, $t = T\hat{t}$ and $p = P\hat{p}$, where $T$
is the \textit{advective} timescale, $T = L/U$ (so that the $\pd{\bs{u}}{t}$ and
$\bs{u}\cdot\grad\bs{u}$ terms are scaled equally). Putting these into the
Navier-Stokes equations, and dropping hats, we get
$$ \frac{\rho U^2}{L} \DDt{\bs{u}} = \frac{-P}{L} \grad P + \frac{\mu U}{L^2} \grad^2 \bs{u}. $$
The Reynolds number is defined as the ratio of the inertia term to the viscous term:
\begin{equation}
 \Rey = \frac{\text{inertia}}{\text{viscosity}}  
      = \frac{\rho U^2 / L}{\mu U / L^2} 
      = \frac{\rho L U}{\mu} 
      = \frac{LU}{\nu}.
\end{equation}

We can proceed in two different ways:
\begin{itemize}
\item Dividing by the viscous scale gives us
$$ \Rey\DDt{\bs{u}} = \frac{-PL}{\mu U} \grad P + \grad^2\bs{u}. $$
For \textit{small} $\Rey$, we choose the viscous pressure scale $P\sim\frac{\mu U}{L}$, 
and we get the Stokes equations 
$$ \grad p = \grad^2 \bs{u} $$
which hold at low Reynolds numbers (i.e. for viscous flows).

\item Alternatively, dividing by the inertia scale gives us
$$ \DDt{\bs{u}} = \frac{-P}{\rho U^2} \grad P + \frac{1}{\Rey} \grad^2\bs{u}. $$
For \textit{large} $\Rey$, we choose the inviscid pressure scale $P\sim\rho U^2$,
to get the inviscid Euler equations 
$$ \DDt{\bs{u}} = -\grad P. $$
\end{itemize}

\section{Exercises}

\subsection{Exercise 1} 

A simple model of two competing species eating the same food takes the form
\begin{align*}
 \dod{N_1}{t} &= r_1 N_1 \left( 1 - \frac{N_1}{K_1} - b_{12} \frac{N_2}{K_2} \right), \\
 \dod{N_2}{t} &= r_2 N_2 \left( 1 - \frac{N_2}{K_2} - b_{21} \frac{N_1}{K_1} \right), \\
\end{align*}
where $N_1$ and $N_2$ are the population sizes. Rescale the equations to
simplify them, and show that the solutions depend only on $\rho = r_2/r_1$,
$b_{12}$ and $b_{21}$. 

\subsection{Exercise 2}

%A bistable system with diffusion is given by
%$$ \dpd{u}{t} = \dpd[2]{u}{x} - u(u-r)(u-1) $$
%where $0<r<1$. Seek a travelling wave solution by setting $\xi = x-ct$ and
%$u(x,t)=f(\xi)$, and find the differential equation satisfied by $f$. 

The concentration of a chemical $C(x,t)$ satisfies the nonlinear diffusion
equation
$$ \dpd{C}{t} = \dpd{}{x}\left(D(C)\dpd{C}{x}\right) $$
and the condition $\int_{-\infty}^\infty C(x,t) \dif x = M$. The diffusivity is
given by $D(C) = kC^p$, and $M$, $k$ and $p$ are positive constants. 

Use dimensional analysis to find a suitable space-like scale $\xi$ and a
space-independent $\eta$ for the similarity solution of the form 
$$ C(x,t) = \eta F(\xi). $$
Use this form to seek the solution initially localised to the origin, and show
that $F$ is of the form
$$ F(\xi) = \begin{cases}
 \left( A - \frac{p}{2(2+p)} \xi^2 \right)^{1/p} & \text{for $|\xi| < \xi_0$} \\
 0 & \text{otherwise}
\end{cases} $$
for some $A$ and $\xi_0$. For the case when $p=2$, find $A$ and $\xi_0$. 

\subsection{Exercise 3: Flow in a 2D thin layer}

Consider a flow in a 2D domain where $x\sim L$ and $y\sim\delta L$, where
$\delta\ll1$ so the domain is thin. How do $\pd{}{x}$ and $\pd{}{y}$ scale?

If $u\sim U$, explain why $v\sim\delta U$, and explain why the advective timescale $T$ is proportional to $L/U$. 

Rescale the Navier-Stokes equations, taking an advective timescale and choosing
the pressure scale $P$ so that it always balances the $x$-momentum equation.
Show that three regimes are possible, depending on how large $\Rey$ is compared
to $\delta$:
\begin{itemize}
    \item If $\delta^2\Rey \ll 1$ then $P\sim\frac{\mu U}{\delta^2L}$, and
    $$ 0 = -\dpd{p}{x} + \dpd[2]{u}{y} \text{ and } 0 = -\dpd{p}{y}. $$
    This is the \textit{lubrication regime}. 
    \item If $\delta^2\Rey\sim1$ then again $P\sim\frac{\mu U}{\delta^2L}$, but
    now 
    $$ \DDt{u} = -\dpd{p}{x} + \dpd[2]{u}{y} \text{ and } 0 = -\dpd{p}{y}. $$
    These are the \textit{unsteady boundary layer equations}. They represent the
    flow of a low-viscosity fluid in a thin layer near a no-slip boundary; the
    thickness of the boundary layer is controlled by the viscosity of the fluid.
    \item For $\delta^2\Rey\gg1$, one has $P\sim\rho U^2$, and 
    $$ \DDt{u} = -\dpd{p}{x} \text{ and } 0 = -\dpd{p}{y} $$
    This is the \textit{shallow water regime}. This regime can be used to model
    the flow of low-viscosity fluids in chutes, rivers or even oceans (provided
    that horizontal lengths are far greater than the depth).
\end{itemize}

\subsection{Exercise 4: Decay of vorticity}

Writing $\bs{\omega} = \curl\bs{u}$ for the vorticity, and using the identities
$$ (\bs{u}\cdot\grad)\bs{u} = (\curl\bs{u})\cross\bs{u} + \grad\left(\frac{1}{2}|\bs{u}|^2\right) $$
and
$$ \grad^2\bs{u} = \grad(\divg\bs{u}) - \curl(\curl\bs{u}), $$
show that
$$ \dpd{\bs{\omega}}{t} + (\bs{u}\cdot\grad)\bs{\omega} - (\bs{\omega}\cdot\grad)\bs{u} = \nu\grad^2\bs{\omega} $$
provided that body forces are conservative. This is the \textit{vorticity equation}.

Why does the $(\bs{\omega}\cdot\grad)\bs{u}$ term vanish in the 2D case? 

Show that vorticity decays over a lengthscale $L\propto\frac{\nu}{U}$.

\end{document}
